\documentclass{article}

\author{Author: Jeremy Greenburg \\ Mentor: Dr. Alton Coalter \\ Second Reader: Joshua Guerin}
\title{The God Core \\ A Science Fiction Video Game Developed in C++}

% For importint code and text files
\usepackage{listings}
% For enumerating the Table of Contents
\usepackage{enumitem}
% Incase I need pictures
\usepackage{graphicx}

%Mathematics Galore
\usepackage{amsmath}
\usepackage{amsfonts}
\usepackage{amssymb}
\usepackage{gensymb}

\usepackage{fullpage}

% Subsection numbering
\setcounter{secnumdepth}{3}

% Only Subsections in Table of Contents
\setcounter{tocdepth}{2}

\begin{document}
\maketitle
\pagebreak

\tableofcontents

\pagebreak

\section{Preamble}

\section{Programming}

\subsection{The Language}

\subsection{APIs}

\subsubsection{OpenGL}
OpenGL, or the Open Graphics Library, is one of the most popular graphics libraries out there. It gives access to linear algebra functions for matrix manipulation (which is important, as 3D graphics relies heavily upon matrix transformations), keyboard and mouse input, and windowing. I chose to use OpenGL over a different graphics library, such as Microsoft's DirectX, because it is open source and cross platform, which would make porting my game to a different Operating System a much easier task if I ever decide to in the future.

\subsubsection{SOIL}
SOIL, or the Simple OpenGL Interface Library, is a small extension to OpenGL that I picked up along the way. It is a texture library that can load .jpg and .png images and bind them to an OpenGL texture, making it very simple to incorporate such images into my game.

\subsubsection{FMOD}
I chose FMOD as the base for my game's audio as it is a simple, lightweight, and free to use sound API, and most other audio APIs that I looked at lacked support for MP3 files. 

\subsubsection{SQlite}
I decided to use SQLite for my database because it is a lightweight simplified version of a SQL database, allowing the game data to be stored and embedded in the application without taking much room or take a great deal of time to perform a query.

\subsection{Game Engine}

I crafted the engine of my game in C++ over two years starting in my second semester sophomore year and ending the first semester of my senior year.

\subsubsection{Camera Control}
The CameraControl class is designed to control and manipulate the player's perspective as they navigate through the game. It contains two ordered triples of floating point numbers: The xyz location of the player, and the rotation along the x axis (looking left/right), the y axis (up/down), and the z axis (barrel roll). It also contains two additional floating point values, the movement speed and the turning speed. 

The player can move forwards and backwards, as well as strafe left and right. To correctly formulate the player's movement, I had to envision a circle centered on the player with a radius of the player's movement speed. Based on the angle from the x and z rotation, the next place that the player move is simply a spot on the circumference of the circle based on the rotation angle, and moving forward can be derived from this formula:

z := z $\pm$ moveSpeed * cos(radian(x\_angle))

x := x $\mp$ moveSpeed * sin(radian(x\_angle))

Following that formula, it's simple to implement movement to the left, right by adding or subtracting 90$\degree$, and backwards movement by adding 180$\degree$.

Whenever OpenGL renders a new frame, the 'camera' is always returned to the origin of the map, so after drawing the level and before flushing the buffer, the Camera Control calls glTranslate to move the camera to the correct location, and then calls glRotate 3 times, once for each axis, to orient the camera in the correct direction.

\subsubsection{Heads Up Display}
The Heads Up Display is drawn after the level is draw, so that it overlays information to the player. It primarily is used to add a bit of flavor to the game by drawing the helmet for the player, but it also serves to display the developer console when activated.

The display also delivers a prompt to the user whenever they are in range of an object that can be interacted with.

\subsubsection{Rectangles and Triangles}
Rectangles and triangles are the two fundamental polygons that build up my game. Rectangles in particular make up the walls, floors, ceilings, doors, terminals, and most of the HUD and menu. They started as simply two arrays- one that holds all 9 (for triangles) or 12 (for rectangles) values describing the coordinates in the game that they inhabit, as well as a 4 value vector containing the objects RGBA values.

For collision purposes, when a rectangle class is expanded with the ability to calculate and store its norm and Plane equation (Form $ax + by + cz + d = 0$).

This equation is calculated using the any three corners of the rectangle (Calling them A, B, and C) as follows:

\noindent
$
\vec{AB} = \left| \begin{array}{c}
	Bx - Ax \\
	By - Ay \\
	Bz - Az
 \end{array} \right|
\vec{AC} = \left| \begin{array}{c}
	Cx - Ax \\
	Cy - Ay \\
	Cz - Az
 \end{array} \right| \\
a = \vec{AB}_2 * \vec{AC}_3 - \vec{AB}_3 * \vec{AC}_2 \\
b = \vec{AB}_3 * \vec{AC}_1 - \vec{AB}_1 * \vec{AC}_3 \\
c = \vec{AB}_1 * \vec{AC}_2 - \vec{AB}_2 * \vec{AC}_1 \\
d = aAx + bAy + cAz
$

The norm of the plane can then be derived using the equation $\sqrt{a^2 + b^2 + c^2}$

\subsubsection{2D}

\subsubsection{Powered Objects}

\subsubsection{Collision Engine}
This determines when the player has collided with an object in the world. There are two types of collisions: player-object collisions and player-wall collisions.

Player object collisions are simple to detect, as both the player and the object can be placed within imaginary "bounding spheres" that extend around the player and object. Collision can be detected with this formula:
$\sqrt{(x_2 - x_1) + (y_2 - y_1) + (z_2 - z_1)} < r_2 + r_1$
If the distance between the two spheres is less than the sum of the radii of the two spheres, the they must be colliding.

Player-wall collisions were much harder to reconcile. Because walls tend to be long and thin, you can't simply place one within a bounding sphere, the resulting sphere would simply be too massive.

To rectify that, the collision is split into two phases: broad and narrow.

In the broad phase, we use the plane equation $ax + by + cz + d$ that is derived in the Rectangle section. We use the formula $\frac{ax + by + cz + d}{\sqrt{a^2 + b^2 + c^2}}$, where x, y, and z are the player's x, y, and z coordinates. If the resulting value is less than the radius of the player's bounding sphere, the player has hit that plane and we move onto the narrow phase.

In the narrow phase, each wall is aligned on an axis: x, y, or z. We simply take the largest and smallest values of the coordinates on that axis (for instance, if the wall is x aligned, we take the largest and smallest x value). If the sphere is in between the two values, the player has hit the wall. Otherwise, they hit the plane but not the wall.

\subsubsection{MusicManager}

\subsubsection{TextEngine}

\subsubsection{SaveManager}

\subsubsection{Keyboard}

\subsubsection{Level Loading}

\subsubsection{Console and Logging}

\section{Appendices}

\subsection{Source Code}

\subsubsection{main.cpp}
	\lstinputlisting[
				basicstyle=\ttfamily \small,
				numbers=left,
				breaklines=true,
 				linerange={1-1000},
 				firstnumber = 1]{../main.cpp}
 				
\subsubsection{CameraControl.h}
	\lstinputlisting[
					basicstyle=\ttfamily \small,
					numbers=left,
					breaklines=true,
					linerange={1-1000},
					firstnumber = 1]{../CameraControl.h}

\subsubsection{CameraControl.cpp}
	\lstinputlisting[
					basicstyle=\ttfamily \small,
					numbers=left,
					breaklines=true,
					linerange={1-1000},
					firstnumber = 1]{../CameraControl.cpp}
					
\subsubsection{CollisionEngine.h}
	\lstinputlisting[
					basicstyle=\ttfamily \small,
					numbers=left,
					breaklines=true,
					linerange={1-1000},
					firstnumber = 1]{../CollisionEngine.h}
					
\subsubsection{CollisionEngine.cpp}
	\lstinputlisting[
					basicstyle=\ttfamily \small,
					numbers=left,
					breaklines=true,
					linerange={1-1000},
					firstnumber = 1]{../CollisionEngine.cpp}
					
\subsubsection{Console.h}
	\lstinputlisting[
					basicstyle=\ttfamily \small,
					numbers=left,
					breaklines=true,
					linerange={1-1000},
					firstnumber = 1]{../Console.h}
					
\subsubsection{Console.cpp}
	\lstinputlisting[
					basicstyle=\ttfamily \small,
					numbers=left,
					breaklines=true,
					linerange={1-1000},
					firstnumber = 1]{../Console.cpp}
					
\subsubsection{Cylinder.h}
	\lstinputlisting[
					basicstyle=\ttfamily \small,
					numbers=left,
					breaklines=true,
					linerange={1-1000},
					firstnumber = 1]{../Cylinder.h}
					
\subsubsection{Cylinder.cpp}
	\lstinputlisting[
					basicstyle=\ttfamily \small,
					numbers=left,
					breaklines=true,
					linerange={1-1000},
					firstnumber = 1]{../Cylinder.cpp}
					
\subsubsection{Door.h}
	\lstinputlisting[
					basicstyle=\ttfamily \small,
					numbers=left,
					breaklines=true,
					linerange={1-1000},
					firstnumber = 1]{../Door.h}

\subsubsection{Door.cpp}
	\lstinputlisting[
					basicstyle=\ttfamily \small,
					numbers=left,
					breaklines=true,
					linerange={1-1000},
					firstnumber = 1]{../Door.cpp}
 				
\subsubsection{GameManager.h}
	\lstinputlisting[
					basicstyle=\ttfamily \small,
					numbers=left,
					breaklines=true,
	 				linerange={1-1000},
	 				firstnumber = 1]{../GameManager.h}

\subsubsection{GameManager.cpp}
	\lstinputlisting[
					basicstyle=\ttfamily \small,
					numbers=left,
					breaklines=true,
	 				linerange={1-1000},
	 				firstnumber = 1]{../GameManager.cpp}
	 				
\subsubsection{GCTypes.h}
	\lstinputlisting[
	 				basicstyle=\ttfamily \small,
	 				numbers=left,
	 				breaklines=true,
	 				linerange={1-1000},
	 				firstnumber = 1]{../GCTypes.h}
	 				
\subsubsection{Globals.h}
	 \lstinputlisting[
	 				basicstyle=\ttfamily \small,
	 				numbers=left,
	 				breaklines=true,
	 				linerange={1-1000},
	 				firstnumber = 1]{../Globals.h}
	 				
\subsubsection{Globals.cpp}
	 \lstinputlisting[
	 				basicstyle=\ttfamily \small,
	 				numbers=left,
	 				breaklines=true,
	 				linerange={1-1000},
	 				firstnumber = 1]{../Globals.cpp}
	 				
	 				
\subsubsection{HeadsUpDisplay.h}
	 \lstinputlisting[
					basicstyle=\ttfamily \small,
	 				numbers=left,
	 				breaklines=true,
	 				linerange={1-1000},
	 				firstnumber = 1]{../HeadsUpDisplay.h}
	 				
\subsubsection{HeadsUpDiplay.cpp}
	 \lstinputlisting[
					basicstyle=\ttfamily \small,
	 				numbers=left,
	 				breaklines=true,
	 				linerange={1-1000},
	 				firstnumber = 1]{../HeadsUpDisplay.cpp}		
	 				
\subsubsection{Keyboard.h}
	 \lstinputlisting[
					basicstyle=\ttfamily \small,
	 				numbers=left,
	 				breaklines=true,
	 				linerange={1-1000},
	 				firstnumber = 1]{../Keyboard.h}
	 				
\subsubsection{Keyboard.cpp}
	 \lstinputlisting[
					basicstyle=\ttfamily \small,
	 				numbers=left,
	 				breaklines=true,
	 				linerange={1-1000},
	 				firstnumber = 1]{../Keyboard.cpp}
	 				
\subsubsection{Level.h}
	 \lstinputlisting[
					basicstyle=\ttfamily \small,
	 				numbers=left,
	 				breaklines=true,
	 				linerange={1-1000},
	 				firstnumber = 1]{../Level.h}
	 				
\subsubsection{Level.cpp}
	 \lstinputlisting[
					basicstyle=\ttfamily \small,
	 				numbers=left,
	 				breaklines=true,
	 				linerange={1-1000},
	 				firstnumber = 1]{../Level.cpp}
	 				
\subsubsection{Logger.h}
	 \lstinputlisting[
	 				basicstyle=\ttfamily \small,
	 				numbers=left,
	 				breaklines=true,
	 				linerange={1-1000},
	 				firstnumber = 1]{../Logger.h}
	 				
\subsubsection{Logger.cpp}
	\lstinputlisting[
	 				basicstyle=\ttfamily \small,
	 				numbers=left,
	 				breaklines=true,
	 				linerange={1-1000},
	 				firstnumber = 1]{../Logger.cpp}
	 				
\subsubsection{MainMenu.h}
	\lstinputlisting[
					basicstyle=\ttfamily \small,
					numbers=left,
					breaklines=true,
					linerange={1-1000},
					firstnumber = 1]{../MainMenu.h}

\subsubsection{MainMenu.cpp}
	\lstinputlisting[
					basicstyle=\ttfamily \small,
					numbers=left,
					breaklines=true,
					linerange={1-1000},
					firstnumber = 1]{../MainMenu.cpp}	 
 				
\subsubsection{MusicManager.h}
	\lstinputlisting[
					basicstyle=\ttfamily \small,
					numbers=left,
					breaklines=true,
	 				linerange={1-1000},
	 				firstnumber = 1]{../MusicManager.h}
	 				
\subsubsection{MusicManager.cpp}
	\lstinputlisting[
					basicstyle=\ttfamily \small,
					numbers=left,
					breaklines=true,
	 				linerange={1-1000},
	 				firstnumber = 1]{../MusicManager.cpp}
	 				
\subsubsection{PauseScreen.h}
	 \lstinputlisting[
					basicstyle=\ttfamily \small,
	 				numbers=left,
	 				breaklines=true,
	 				linerange={1-1000},
	 				firstnumber = 1]{../PauseScreen.h}
	 				
\subsubsection{PauseScreen.cpp}	
	 \lstinputlisting[
					basicstyle=\ttfamily \small,
	 				numbers=left,
	 				breaklines=true,
	 				linerange={1-1000},
	 				firstnumber = 1]{../PauseScreen.cpp}
	 				
\subsubsection{Plane.h}
	\lstinputlisting[
					basicstyle=\ttfamily \small,
	 				numbers=left,
	 				breaklines=true,
	 				linerange={1-1000},
	 				firstnumber = 1]{../Plane.h}
	 				
\subsubsection{Plane.cpp}
	\lstinputlisting[
					basicstyle=\ttfamily \small,
	 				numbers=left,
	 				breaklines=true,
	 				linerange={1-1000},
	 				firstnumber = 1]{../Plane.cpp}
	 			
\subsubsection{Return.h}
	\lstinputlisting[
	 				basicstyle=\ttfamily \small,
	 				numbers=left,
	 				breaklines=true,
	 				linerange={1-1000},
	 				firstnumber = 1]{../Return.h}
	 				
\subsubsection{Resource.h}
	\lstinputlisting[
					basicstyle=\ttfamily \small,
					numbers=left,
					breaklines=true,
					linerange={1-1000},
					firstnumber = 1]{../return.h}
	 				
\subsubsection{SaveManager.h}
	\lstinputlisting[
					basicstyle=\ttfamily \small,
					numbers=left,
					breaklines=true,
	 				linerange={1-1000},
	 				firstnumber = 1]{../SaveManager.h}
	 				
\subsubsection{SaveManager.cpp}
	\lstinputlisting[
					basicstyle=\ttfamily \small,
					numbers=left,
					breaklines=true,
	 				linerange={1-1000},
	 				firstnumber = 1]{../SaveManager.cpp}
	 				
\subsubsection{Switch.h}
	\lstinputlisting[
	 				basicstyle=\ttfamily \small,
	 				numbers=left,
	 				breaklines=true,
	 				linerange={1-1000},
	 				firstnumber = 1]{../Switch.h}
	 				
\subsubsection{Switch.cpp}
	\lstinputlisting[
	 				basicstyle=\ttfamily \small,
	 				numbers=left,
	 				breaklines=true,
	 				linerange={1-1000},
	 				firstnumber = 1]{../Switch.cpp}
	 				
\subsubsection{Terminal.h}
	\lstinputlisting[
					basicstyle=\ttfamily \small,
	 				numbers=left,
	 				breaklines=true,
	 				linerange={1-1000},
	 				firstnumber = 1]{../Terminal.h}
	 				
\subsubsection{Terminal.cpp}	
	\lstinputlisting[
					basicstyle=\ttfamily \small,
	 				numbers=left,
	 				breaklines=true,
	 				linerange={1-1000},
	 				firstnumber = 1]{../Terminal.cpp}
	 				
\subsubsection{TextEngine.h}
	\lstinputlisting[
					basicstyle=\ttfamily \small,
	 				numbers=left,
	 				breaklines=true,
	 				linerange={1-1000},
	 				firstnumber = 1]{../TextEngine.h}
	 				
\subsubsection{TextEngine.cpp}
	\lstinputlisting[
					basicstyle=\ttfamily \small,
	 				numbers=left,
	 				breaklines=true,
	 				linerange={1-1000},
	 				firstnumber = 1]{../TextEngine.cpp}
	 				
\subsubsection{Triangle.h}
	\lstinputlisting[
					basicstyle=\ttfamily \small,
	 				numbers=left,
	 				breaklines=true,
	 				linerange={1-1000},
	 				firstnumber = 1]{../Triangle.h}
	 				
\subsubsection{Triangle.cpp}
	\lstinputlisting[
					basicstyle=\ttfamily \small,
	 				numbers=left,
	 				breaklines=true,
	 				linerange={1-1000},
	 				firstnumber = 1]{../Triangle.cpp}
	 				
	 				
\subsubsection{Trigger.h}
	\lstinputlisting[
	 				basicstyle=\ttfamily \small,
	 				numbers=left,
	 				breaklines=true,
	 				linerange={1-1000},
	 				firstnumber = 1]{../Trigger.h}
	 				
\subsubsection{Trigger.cpp}
	\lstinputlisting[
	 				basicstyle=\ttfamily \small,
	 				numbers=left,
	 				breaklines=true,
	 				linerange={1-1000},
	 				firstnumber = 1]{../Trigger.cpp}
	 				
\subsubsection{Triple.h}
	\lstinputlisting[
	 				basicstyle=\ttfamily \small,
	 				numbers=left,
	 				breaklines=true,
	 				linerange={1-1000},
	 				firstnumber = 1]{../Triple.h}
	 				
\subsubsection{Triple.cpp}
	\lstinputlisting[
	 				basicstyle=\ttfamily \small,
	 				numbers=left,
	 				breaklines=true,
	 				linerange={1-1000},
	 				firstnumber = 1]{../Triple.cpp}
	 				
\subsubsection{TwoD.h}
	\lstinputlisting[
					basicstyle=\ttfamily \small,
					numbers=left,
					breaklines=true,
					linerange={1-1000},
					firstnumber = 1]{../TwoD.h}

\subsubsection{TwoD.cpp}	
	\lstinputlisting[
					basicstyle=\ttfamily \small,
					numbers=left,
					breaklines=true,
					linerange={1-1000},
					firstnumber = 1]{../TwoD.cpp}

\subsection{Database}

\subsubsection{Walls}
	\includegraphics[width=18cm]{WALLS}

\subsubsection{Doors}

\subsubsection{Switches}

\subsubsection{Terminals}

\subsubsection{Triggers}

\subsection{Images}

\subsubsection{Main Menu}
	\includegraphics[width=18cm]{../Resources/Images/Main}
\subsubsection{Terminal Banner}
	\includegraphics[width=18cm]{../Resources/Images/banner}
\subsubsection{Game Icon}
	\includegraphics[width=5cm]{../Resources/Images/Core}
\subsection{Music}

\end{document}